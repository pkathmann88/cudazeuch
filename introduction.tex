\section{Einf\"uhrung}

Das hier pr\"asentierte Paper behandelt die im Seminar {\it Distributed Filesystem} aufgetragene Anpassung des bestehenden Systems zur Ermittlung von Disparit\"ats-Karten aus zwei vorliegenden Stereobildern. Das vorgegebene System operiert mittels CUDA und gsoap \"uber ein Netzwerk von Rechnern um die rechenaufw\"andige Aufgabe in kleinere Teilaufgaben zu unterteilen und damit die Rechenzeit insgesamt zu verk\"urzen.
Im Seminar wurde eine Anpassung bzw. Verbesserung des Systems von Seiten der Teilnehmer gefordert, was zentraler Bestandteil dieser Ausarbeitung ist. Unsere Gruppe {\it ``EinCooler Name''} entschied sich daf\"ur, die Daten, die vom Server an die angemeldeten Clienten weitergeleitet werden, nicht im XML-Format zu versenden sondern stattdessen die Bildwerte im das Base64 Format zu verschicken und vor ort wieder in Bildinformationen zur\"uck zu konvertieren. Als erhofftes Ergebnis war eine Laufzeitverringerung erhofft, die durch das Aussetzen einer aufwendigen XML-Konvertierung auftreten sollte.\\\\
Das Nachfolgenden Paper beschreibt dies in den folgenden Kapiteln. Im Kapitel Aufbau wird auf den Kontext der Entwicklung und der anschlie{\"ss}enden Tests eingegangen. Daraufhin folgt im Kapitel Implementierung eine Beschreibung vom programmiertechnisch Aufwand und Entscheidungen. Es folgt eine Beschreibung des Testverlaufs und der Ergebnisse im Kapitel Durchf\"uhrung. Das Paper schlie{\ss}st mit dem Kapitel Fazit und beschreibt auf Basis der Testerwerte das Ergebnis dieser Ausarbeitung und ob die urspr\"ungliche Idee zu einem Erfolg f\"uhrte.

\section{Aufbau}

Im folgenden Abschnitt findet eine Beschreibung der verwendeten Hardware und des allgemeinen Testfalls statt.\\

Entwickelt und getestet wurde das System an drei Rechnern des G40 CIP-Pools mit identischen Spezifikationen. Es waren Rechner mit einem Intel Core i7 CPU mit 8 x 4 GHz. Eine Nvidia Geforce GTX 1080 war als Grafikprozessor in Jedem eingebaut. Das Betriebssystem war ein 64bit Ubuntu 16.04 LTS mit 31,2 GiB Arbeitsspeicher.\\\\
F\"ur jeden Testdurchlauf wurde, zum Zweck der Vergleichbarkeit, das gleiche Testbild bzw. das gleiche Testbild-Paar von Stereobildern benutzt. Die Wahl fiel auf das Bild "Horse". Zu Beginn folgten diverse Testdurchl\"aufe mit dem gegebenen System um Grunddaten zu erhalten. Bei jedem Durchlauf wurden verschiedene Parameter-Kombinationen eingestellt, um ihre Wirkung auf die Laufzeit zu erfassen. Ge\"andert wurden bei jedem Durchlauf die Parameter Tau und die Skalierung. Betrachtet und notiert wurde bei den ausgegebenen Ergebnissen nur die Zeit.\\\\
Es folgten die Testdurchl\"aufe mit dem angepassten System mit der Base64 Encodierung. F\"ur jeden Durchlauf wurden die gleichen Einstellungen f\"ur die Parameter genutzt wie bei dem ersten Durchlauf, um auch hier den Einfluss der Parameter auf die Zeit zu erfassen. Die ben\"otigte Zeit des Systems wurde pro Durchlauf erfasst und mit dem des urspr\"unglichen Systems verglichen.