\section{Einf\"uhrung}

Das hier pr\"asentierte Paper behandelt die im Seminar {\it Distributed Filesystem} aufgetragene Anpassung des bestehenden Systems zur Ermittlung von Disparit\"ats-Karten aus zwei vorliegenden Stereobildern. Das vorgegebene System operiert mittels CUDA und gsoap \"uber ein Netzwerk von Rechnern um die rechenaufw\"andige Aufgabe in kleinere Teilaufgaben zu unterteilen und damit die Rechenzeit insgesamt zu verk\"urzen.
Im Seminar wurde eine Anpassung bzw. Verbesserung des Systems von Seiten der Teilnehmer gefordert, was zentraler Bestandteil dieser Ausarbeitung ist. Unsere Gruppe {\it ``Ein Cooler Name''} entschied sich daf\"ur, die Daten, die vom Server an die angemeldeten Clienten weitergeleitet werden, nicht im XML-Format zu versenden sondern stattdessen die Bildwerte im das Base64 Format zu verschicken und vor ort wieder in Bildinformationen zur\"uck zu konvertieren. Als erhofftes Ergebnis war eine Laufzeitverringerung erhofft, die durch das Aussetzen einer aufwendigen XML-Konvertierung auftreten sollte.\\\\
Das Nachfolgenden Paper beschreibt dies in den folgenden Kapiteln. Im Kapitel ``Implementierung'' erfolgt eine Beschreibung vom programmiertechnisch Aufwand und Entscheidungen. Daraufhin folgt eine Beschreibung des Testverlaufs und der Ergebnisse im Kapitel ``Auswertung''. Das Paper schlie{\ss}st mit dem einem Fazit und beschreibt auf Basis der Testerwerte das Ergebnis dieser Ausarbeitung und ob die urspr\"ungliche Idee zu einem Erfolg f\"uhrte unter dem Punkt ``Messergebnisse''.