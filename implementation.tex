\section{Implementierung}
Im folgenden Kapitel wird die Implementierung der Schnittstelle n\"aher betrachtet. 
Zun\"achst wird die bestehende Schnittstelle analysiert und es wird definiert, 
welche \"Anderungen durchzuf\"uhren sind. Anschließend wird die Implementierung 
der \"Anderungen n\"aher untersucht, bevor im folgenden Kapitel die Ergebnisse betrachtet 
und ausgewertet werden.

\subsection{Ausgangslage}
Wie in der Einleitung beschrieben, sollte eine bestehende Softwarel\"osung abge\"andert werden.
In der bestehenden L\"osung wurde die verteilte Ausf\"uhrung der CUDA-Berechnungen \"uber SOAP-Requests 
eingeleitet, welche die Bilddaten als Array von Integern beinhalteten. Die prim\"aren Nachteile
der \"Ubertragung der Bilddaten als Array von Integern liegen darin, dass zum Einen ein erh\"ohter
Netwerk-Traffic anf\"allt, zum Anderen auch ein nicht unerheblicher Aufwand durch das Auswerten
der XML-Daten entsteht.\\
Der erh\"ohte Netzwerk-Traffic l\"asst sich auf die Struktur von XML-Dokumenten
zur\"uckf\"uhren, welche pro Datum ein \"offnendes und ein schließendes Tag ben\"otigen. Da es sich bei den 
Daten um ein Graustufen handelt, werden Integer \"ubertragen, die im Allgemeinen nicht mehr als drei
Stellen beinhalten. In diesem Fall bestehen die Tags jeweils aus mehr Zeichen als der zu \"ubertragende
Wert. Hierdurch entsteht ein nicht zu vern\"achl\"assigender Overhead, welcher bei einer \"Ubertragung als
Base64-Codiertes Datum minimiert wird, da die Tags nur ein mal pro Bild \"ubertragen werden m\"ussen.\\
Der Aufwand, welcher durch die Auswertung der XML-Daten entsteht, ist daher nicht unerheblich, da zum einen
die Daten als Strings vorliegen, welche in Integer konvertiert werden m\"ussen. Zum anderen jedoch bei 
SOAP-Schnittstellen im Allgemeinen auch eine Validierung der Daten vorgenommen wird. Aufgrunddessen sollte
klar sein, warum die \"Ubertragung der Bilder als Arrays einzelner Integer einen höheren Aufwand erzeugt, als die 
\"Ubertragung zweier Base64-Codierter Strings.

\subsection{Durchf\"uhrung}
In diesem Abschnitt wird die Durchf\"uhrung der im vorherigen Kapitel beschriebenen \"Anderungen erkl\"art.
Die vorzunehmenden \"Anderungen lassen sich in zwei Teilschritte unterteilen. Zum einen die Anpassung der
SOAP-Schnittstelle und zum anderen die Konvertierung der Bilddaten in Base64.\\
Um die SOAP-Schnittstelle anzupassen wurde die WSDL dahingehend geändert, dass die Bilder nun nicht mehr als 
Arrays einzelner Pixel \"ubertragen werden, sondern als einzelner String. Auf Basis der ge\"anderten WSDL wurden
unter Verwendung des Tools gsoap Client- und Serverseitig Code erzeugt, welcher die Abhandlung des SOAP-Prozesses
erm\"oglicht.\\
Um die Bilder als String zu \"ubertragen, m\"ussen diese noch in das Base64-Format konvertiert werden. Hierzu 
wurde eine einfach Konvertierungsfunktion implementiert. Diese Funktion liefert die bin\"aren Eingabedaten als
Base64-String zur\"uck. Bei der \"Ubertragung wird dieser String zusammen mit der Gr\"o{\ss}e des Originalbildes
versandt. Die Serverseite muss nun auf Basis der Originalgr\"o{\ss}e die Gr\"o{\ss}e der Daten als Base64-String 
errechnen und kann anschlie{\ss}end unter Zuhilfenahme der neu errechneten Base64-Gr\"o{\ss}e wieder ein Bild erzeugen.
Diese Bilder werden dann miteinander verglichen und die Ergebnisse werden weiterhin als Array von Integern an den 
Client zur\"uck gesandt.
